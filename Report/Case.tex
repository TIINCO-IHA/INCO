\documentclass[Main]{subfiles}

\begin{document}
\section{Case Study}

%(Præsentation af opgaven)
%
%(Implementering i matlab)
%
%(Kodeeksempler)
%
%(Problemer undervejs --> faldgruber)

\subsection{Case Study}
This case study concerns the implementation of a Meggitt decoder which has been implemented in Matlab. 

\noindent The Meggitt decoder is implemented as a function that can decode cyclic codes based on the Meggitt algorithm. Besides the Meggitt decoder function the Matlab codes consist of two other functions: a cyclic code encoder, and a main function.

\noindent The cyclic code encoder encodes a message vector into systematic code. The main function generates a message vector and then uses the cyclic code encoder to encode the message vector. It then transmits the code vector through an artificial communication channel and hereby introduces errors to the code vector. Finally it uses the Meggitt decoder to decode the received vector. 

\noindent The conditions given for the case study are: 
\begin{itemize} \itemsep0pt \parskip0pt \parsep0pt
\item Code length: 15
\item Message length: 7
\item Generator polynomial: $g(X)=1+X^4+X^6+X^7+X^8$
\end{itemize} 

\noindent All three functions are described in detail in the following. 


\subsection{Implementation of Encoder}
The implemented encoder is a Matlab function, $cyclicEncoder$. The encoder is able to encode a message vector into systematic code by receiving a generator polynomial $g$ and a message vector $m$. In addition the encoder has been made generic so that it also is possible to state the code length $n$ and the message length $k$ - if nothing is stated, the default value of n is 15 and k is 7. The output of the function is a code vector, $c$. 

\noindent Initially the function converts the message polynomial in vector form into a vector of the length n by shifting it $n-k$ to the right and thereby forming the polynomial $X^{n-k}m(X)$. 

\noindent Secondly, the now right-shifted message polynomial is divided by the generator polynomial to find the remainder polynomial, $p(X)$.

\noindent Finally, adding the right-shifted message polynomial and the remainder polynomial results in the code polynomial.  

\begin{lstlisting}[caption=Cyclic Encoder, style=Code-Matlab, label=lst:refID]
 % cyclicEncoding is a function that encodes a message vector into systematic code given the following parameters
 % g is the generator polynomial in vector form
 % m is message vector
 % n is the code length
 % k is the message length
 function c = cyclicEncoding(g,m,n,k)

 % Form the polynomial X^(n-k)*m(X) by right-shifting m n-k times 
 shifted_m = [zeros(1,n-k) m];

 % Find the remainder, p, by dividing shifted_m by g over GF(2) (q is a factor)
 [q, p] = gfdeconv(shifted_m,g,2);

 % Extend p with zeros to having the length of n (to be able to add p with shifted_m)
 p = [p zeros(1, n-length(p))];

 % Obtain the code vector c by using binary addition to add shifted_m and p 
 c = mod(shifted_m+p,2);
\end{lstlisting}

\subsection{Implementation of Main Function}
The main function, $miniproject$, can correct up to two errors where the errors can be specified in numbers and location.

The first parameter is the error-count which specifies how many errors the function should correct.
The main function can also be called without any parameters and will set the default error-count to 2   

\begin{lstlisting}[caption=Main Function, style=Code-Matlab, label=lst:refID]
function test = miniproject(t_errors, errorloc)
% function test = miniproject(t_errors, errorloc)
% t_errors:       Number of errors to be introduced. Maximum 2. Default 2.
% errorloc:       Location of the t_errors errors. Should be a vector with
%                 scalars between 1-15 corresponding to bit location 1-15 (degree 0-14).
% test:           Returns one if the decoded message, c, is equal to the input
%                 message, r.
addpath('../Encoder');
addpath('../Meggitt');

% Generate message vector
n = 15;
k = 7;
% g(x) = 1+X^4+X^6+X^7+X^8
g = [1 0 0 0 1 0 1 1 1];

if(nargin < 1)  % no input parameters
    t = 2;
    errorlocation = sort(randi(n,1,t));
elseif(nargin > 0)
    t = t_errors;
    if(nargin < 2)
        errorlocation = sort(randi(n,1,t));
    else
        errorlocation = sort(errorloc);
    end
end

% generate random numbers in GF(2)
m = mod(randi(2,1,k),2);

% Encode message vector by the ciclic code encoder
c = cyclicEncoding(g,m,n,k);

while(t > 2 && length(unique(errorlocation)) ~= t)
    errorlocation = sort(randi(n,1,t));
    disp('Errorlocation was recalculated!');
end
r = c;

% Flip bit at generated error location i the received vector.
% Add 1 and modulo 2 for binary addition
r(errorlocation) = mod(c(errorlocation)+1,2);

% Decode the received vector by the meggitt decoder

[e, r_c, tag] = meggitt_decoder(r, g, n, k);

if(isequal(c, r_c) == 0)
    disp('The codeword was not decoded correct');
else
    disp('The codeword was successfully decoded');
end

c_t = mat2str(c);
e_t = mat2str(e);
r_t = mat2str(r);
rc_t = mat2str(r_c);
t_out = 'The coded vector is    c = %s.\nAfter transmission is  r = %s.\nThe error vector is    e = %s.\nThe corrected vector r_c = %s\nThe tag from the decoder claims: %s.\n';
fprintf(t_out,c_t,r_t,e_t,rc_t,tag);
% clear c_t; clear e_t; clear r_t; clear rc_t; clear t_out;

save variables
test = isequal(c, r_c);
\end{lstlisting}

\end{document}
