\documentclass[Main]{subfiles}

\begin{document}
\section{Case Study}

%(Præsentation af opgaven)
%
%(Implementering i matlab)
%
%(Kodeeksempler)
%
%(Problemer undervejs --> faldgruber)

This case study concerns the implementation of a Meggitt decoder which has been implemented in \texttt{Matlab}. 

The Meggitt decoder is implemented as a function that can decode cyclic codes based on the Meggitt algorithm. Besides the Meggitt decoder function the \texttt{Matlab} codes consist of two other functions: a cyclic code encoder, and a main function.

The cyclic code encoder encodes a message vector into systematic code. The main function generates a message vector and then uses the cyclic code encoder to encode the message vector. It then transmits the code vector through an artificial communication channel and hereby introduces errors to the code vector. Finally it uses the Meggitt decoder to decode the received vector. 

The conditions given for the case study are: 
\begin{itemize} \itemsep0pt \parskip0pt \parsep0pt
\item Code length: 15
\item Message length: 7
\item Generator polynomial: $g(X)=1+X^4+X^6+X^7+X^8$
\end{itemize} 

All three functions are described in detail in the following. 











\subsection{Implementation of Encoder}
\label{sec:ImplementEncoder}
The implemented encoder is a \texttt{Matlab} function, \texttt{cyclicEncoder}. The encoder is able to encode a message vector into systematic code by receiving a generator polynomial $g$ and a message vector $m$. In addition the encoder has been made generic so that it also is possible to state the code length $n$ and the message length $k$ - if nothing is stated, the default value of n is 15 and k is 7. The output of the function is a code vector, $c$. 

Initially the function converts the message polynomial in vector form into a vector of the length n by shifting it $n-k$ to the right and thereby forming the polynomial $X^{n-k}m(X)$. 

Secondly, the now right-shifted message polynomial is divided by the generator polynomial to find the remainder polynomial, $p(X)$.

Finally, adding the right-shifted message polynomial and the remainder polynomial results in the code polynomial, $c$.  

\begin{lstlisting}[caption=Cyclic Encoder, style=Code-Matlab, label=lst:refID]
 % cyclicEncoding is a function that encodes a message vector into systematic code given the following parameters
 % g is the generator polynomial in vector form
 % m is message vector
 % n is the code length
 % k is the message length
 function c = cyclicEncoding(g,m,n,k)

 % Form the polynomial X^(n-k)*m(X) by right-shifting m n-k times 
 shifted_m = [zeros(1,n-k) m];

 % Find the remainder, p, by dividing shifted_m by g over GF(2) (q is a factor)
 [q, p] = gfdeconv(shifted_m,g,2);

 % Extend p with zeros to having the length of n (to be able to add p with shifted_m)
 p = [p zeros(1, n-length(p))];

 % Obtain the code vector c by using binary addition to add shifted_m and p 
 c = mod(shifted_m+p,2);
\end{lstlisting}







\subsection{Implementation of Meggitt decoder}
\label{sec:ImplementMeggittDecoder}
The Meggitt decoder is able to decode an input message, $r(X)$, which may contain 1 or 2 errors and return the correct output vector, $v(X)$.
If $r(X)$ contains 3 errors the decoder should be able to detect the error but not able to correct it.

When calling the Meggitt decoder it first calculates the error pattern as shown in matrix \ref{lst:ErrorPattern}, section \ref{sec:MeggittDedocerMaterial}, and then finds the corresponding syndromes as shown in matrix \ref{lst:syndromeErrorPattern}.

In Matlab this is done with the function in \codeTitle \ref{lst:ErrorPatternCreate}:

\begin{lstlisting}[caption=Error Pattern Creation, style=Code-Matlab, label=lst:ErrorPatternCreate]
 function syndromes_errorpattern = generateErrorPattern(g,n,k)
 % generate syndrome error pattern
 
 errorpattern = fliplr(eye(n));
 errorpattern(:,n) = 1;
 syndromes_errorpattern = zeros(n,n-k);
 
 for i = 1:n
     [q, s] = gfdeconv(errorpattern(i,:), g, 2);
     if(length(s) ~= n-k)
         syndromes_errorpattern(i,:) = [s zeros(1,n-k-length(s))];
     else
         syndromes_errorpattern(i,:) = s;
     end
 end
\end{lstlisting}

Next is the initiation of the syndrome register.
As it was shown in section \ref{sec:Meggitt} the initialization of the syndrome register is the same as making a polynomial division of $S(X)$ and $g(X)$. With this knowledge the function is implemented in \texttt{Matlab} is as shown in \codeTitle \ref{lst:SyndromeInit}.
If the syndrome vector is smaller than $k$, the syndrome is filled up with zeros so they all have a length of $k$.

\begin{lstlisting}[caption=Syndrome Initialization, style=Code-Matlab, label=lst:SyndromeInit]
 function iSyndrome = init_Syndrome(r,g,n,k)
 % function iSyndrome = init_Syndrome(r,g,n,k)
 
 [q, iSyndrome] = gfdeconv(r,g,2);
 
 if(length(iSyndrome) ~= n-k)
     iSyndrome = [iSyndrome zeros(1,n-k-length(iSyndrome))];
 end
\end{lstlisting}

The Meggitt decoder is now ready to decode, and the decoding is shown in \codeTitle \ref{lst:ErrorDetection}.

First the decoder checks whether the syndrome has an error or not, by comparing the syndrome with all the syndromes in the \texttt{syndromes\_errorpattern}.
If there is an error the end bit in the buffer, here $r$, is corrected before it is shifted.
After the shift a new syndrome is calculated and it is ensured, that the length is $k$.

\begin{lstlisting}[caption=Error Detection, style=Code-Matlab, label=lst:ErrorDetection]
 for i = 1:n
     % Test of syndrome is equal to one of the error syndromes
     for b = 1:n
         % Check if the calculated syndrome is equal to one of the rows in syndrome
         % table
         if(isequal(si, syndromes_errorpattern(b,:))==1)
             % correct bit, if error happened
             error_vec(i) = 1;
             r(end) = mod(r(end)+1,2);
         end
     end
     
     % Right shift the received vector 1 bit
     ri = circshift(r, [1 1]);
     % Calculate new syndrome register
     [q, si] = gfdeconv(ri,g,2);
     % Set r to ri (make the shift in memory too)
     r = ri;
     % zeropad si to compare with the syndrome table
     if(length(si) ~= n-k)
         si = [si zeros(1,n-k-length(si))];
     end
 end
\end{lstlisting}

After the decoding the Meggitt decoder sums up all information to create a tag to indicate whether the code is decodable or undecodable but detectable, and returns the error vector and the code vector.

This \texttt{Matlab} code for this is shown in \codeTitle \ref{lst:DecodeReturn}.

\begin{lstlisting}[caption=Meggitt Decoder's Return, style=Code-Matlab, label=lst:DecodeReturn]
 if sum(si)~=0 && sum(error_vec) > 2
     tag = 'The received vector contains more than 2 errors';
 elseif sum(si)==0 && sum(error_vec) <= 2 && sum(error_vec) >= 1
     tag = ['The decoder has corrected ', num2str(sum(error_vec)), ' error(s)'];
 else
     tag = 'The received vector contains more than 2 errors';
 end
 
 error_vec = fliplr(error_vec);
 code_vec = r;
\end{lstlisting}








\subsection{Implementation of Main Function}
The main function generates a message vector, encodes it into a code vector, transmits it and hereby introduces errors in the code vector, and finally decodes the code vector using Meggitt decoding.

The main function makes use of the implementations of the generic cyclic encoder and the Meggitt decoder described in sections \ref{sec:ImplementEncoder} and \ref{sec:ImplementMeggittDecoder}. 
When using the main function it is possible to decide how many errors, $t$, should be introduced during transmission, as well as it is possible to manually introduce the error location(s).

Initially the code length, the message length and the generator polynomial are set. 
A message vector is generated and encoded into a code vector by the cyclic encoder. 
This is shown in \codeTitle \ref{lst:GenerateVector}. 

\begin{lstlisting}[caption=Generating and Encoding a Vector, style=Code-Matlab, label=lst:GenerateVector]
 % Set the code length, message length and the generator polynomial
 n = 15;
 k = 7;
 g = [1 0 0 0 1 0 1 1 1];   % g(x) = 1+X^4+X^6+X^7+X^8

 % Generate a message vector by using random numbers in GF(2)
 m = mod(randi(2,1,k),2);

 % Encode the message vector by the cyclic code encoder function
 c = cyclicEncoding(g,m,n,k);
\end{lstlisting}

Secondly, one or more errors are introduced to the code vector depending on the arguments received by the main function - if no specific number of errors have been stated two errors will be introduced. 
The error locations used will be according to the arguments received by the main function - if no specific error locations have been stated random error location(s) will be generated. 
The code vector is put into the received vector and the bits at the error location(s) in the received vector are flipped.

Finally, the received vector is decoded by the Meggitt decoder and it is verified whether the decoded vector is equal to the code vector. 
The result of whether the decoding was successful or not is printed.

The error handling, decoding and the verification of the decoding is shown in \codeTitle \ref{lst:errorHandling}.

\begin{lstlisting}[caption=Error Handling and Decoding, style=Code-Matlab, label=lst:errorHandling]
 % Transmit the code vector through an artificial communication channel, i.e. introduce at least one error to the code vector:

 % If the number of errors stated to be introduced is zero, introduce 2 errors
 % If no error location has been stated, generate random error locations
 % State the error locations in a sorted array
 if(nargin < 1)  % no input parameter stating number of  errors
     t = 2;
     errorlocation = sort(randi(n,1,t));
 elseif(nargin > 0)
     t = t_errors;
     if(nargin < 2)
         errorlocation = sort(randi(n,1,t));
     else
         errorlocation = sort(errorloc);
     end
 end

 % If there are two or more errors and they happen to be in the same location then generate two new error locations
 while(t > 2 && length(unique(errorlocation)) ~= t)
     errorlocation = sort(randi(n,1,t));
     disp('Errorlocation was recalculated!');
 end

 % Put the code vector c into the received vector r
 r = c;

 % Introduce errors at the generated error location(s) i the received vector
 % by adding 1 and modulo 2 for binary addition (flipping the bits)
 r(errorlocation) = mod(c(errorlocation)+1,2);

 % Decode the received vector by the Meggitt decoder
 [e, r_c, tag] = meggitt_decoder(r, g, n, k);

 % Print whether the decoding was successfull 
 if(isequal(c, r_c) == 0)
     disp('The codeword was not decoded correct');
 else
     disp('The codeword was successfully decoded');
 end
\end{lstlisting}
\end{document}
