\documentclass[Main]{subfiles}

\begin{document}

\subsection{Implementation of Meggitt decoder}
The Meggitt decoder is able to decode an input message, $r(X)$, which may contain 1 or 2 error and return the correct output vector, $v(X)$.
If $r(X)$ contains 3 error it the decoder should be able to detect error but not able to correct it.

When calling the meggitt decoder is start with calculate the error pattern as in \fxnote{HENVISNING meggitt decoder slutning, tabel} and finding the corresponding syndromes.
In Matlab this is done with the function in \codeTitle \ref{lst:ErrorPatternCreate}

\begin{lstlisting}[caption=Error pattern creation, style=Code-Matlab, label=lst:ErrorPatternCreate]
 function syndromes_errorpattern = generateErrorPattern(g,n,k)
 % generate syndrome error pattern
 
 errorpattern = fliplr(eye(n));
 errorpattern(:,n) = 1;
 syndromes_errorpattern = zeros(n,n-k);
 
 for i = 1:n
     [q, s] = gfdeconv(errorpattern(i,:), g, 2);
     if(length(s) ~= n-k)
         syndromes_errorpattern(i,:) = [s zeros(1,n-k-length(s))];
     else
         syndromes_errorpattern(i,:) = s;
     end
 end
\end{lstlisting}

The initiation of the syndrome register is happening next.
As it was shown in \fxnote{HENVISNING TIL AFSNIT MEGGITT DECODER} the initialization of the syndrome register is the same at making a polynomial division of $S(X)$ and $g(X)$.
With this knowledge the function in MatLab is as shown in \codeTitle \ref{lst:SyndromeInit}.
If the syndrome vector is less than $k$, when it don't have any bit in the high end, the syndrome is filled up with zeros so they all have a length of $k$.

\begin{lstlisting}[caption=Syndrome initialization, style=Code-Matlab, label=lst:SyndromeInit]
 function iSyndrome = init_Syndrome(r,g,n,k)
 % function iSyndrome = init_Syndrome(r,g,n,k)
 
 [q, iSyndrome] = gfdeconv(r,g,2);
 
 if(length(iSyndrome) ~= n-k)
     iSyndrome = [iSyndrome zeros(1,n-k-length(iSyndrome))];
 end
\end{lstlisting}

The Meggitt decoder is now ready to decode which is happening in \codeTitle \ref{lst:ErrorDetection}.
Start to check for whether the syndrome has an error or not, by checking the syndrome with all the syndromes in the $syndrome\_errorpattern$.
If it has an error the end bit in the buffer, here $r$, is corrected before it is shifted.
After the shift a new syndrome is calculated and make sure the length is $k$.

\begin{lstlisting}[caption=Error detection, style=Code-Matlab, label=lst:ErrorDetection]
 for i = 1:n
     % Test of syndrome is equal to one of the error syndromes
     for b = 1:n
         % Check if the calculated syndrome is equal to one of the rows in syndrome
         % table
         if(isequal(si, syndromes_errorpattern(b,:))==1)
             % correct bit, if error happened
             error_vec(i) = 1;
             r(end) = mod(r(end)+1,2);
         end
     end
     
     % Right shift the received vector 1 bit
     ri = circshift(r, [1 1]);
     % Calculate new syndrome register
     [q, si] = gfdeconv(ri,g,2);
     % Set r to ri (make the shift in memory too)
     r = ri;
     % zeropad si to compare with the syndrome table
     if(length(si) ~= n-k)
         si = [si zeros(1,n-k-length(si))];
     end
 end
\end{lstlisting}

After the decoding it sum up all information to create the tag and return the error vector and the code vector.
This MatLab code is shown in \codeTitle \ref{lst:DecodeReturn}.

\begin{lstlisting}[caption=Meggitt decode return, style=Code-Matlab, label=lst:DecodeReturn]
 if sum(si)~=0 && sum(error_vec) > 2
     tag = 'The received vector contains more than 2 errors';
 elseif sum(si)==0 && sum(error_vec) <= 2 && sum(error_vec) >= 1
     tag = ['The decoder has corrected ', num2str(sum(error_vec)), ' error(s)'];
 else
     tag = 'The received vector contains more than 2 errors';
 end
 
 error_vec = fliplr(error_vec);
 code_vec = r;
\end{lstlisting}

\end{document}