\documentclass[Main]{subfiles}

\begin{document}

\subsection{Meggitt decoder}

The Meggitt decoder can decode an input message, $r(x)$, which may contain up to 2 errors and return the correct output vector, $v(x)$.\fxnote{Hvorfor hedder det en "vector", når vi vi skriver $r(x)$?}
The Meggitt decoder can be called in Matlab with the function in \codeTitle \ref{lst:Meggitt}

\begin{lstlisting}[caption=Meggit decoder function, style=Code-Matlab, label=lst:Meggitt]
function res = Meggitt(r)

	syndrome = InitSyndrome(r);
	buffer = r;
	L = length(r(1,:));
	
	error = zeros(1, L);
	
	for i = 1:L
	   [error(1,i), syndrome] = Detector(0, syndrome); 
	
	   output = mod(buffer(1,L) + error(1,i),2);
	
	   buffer = circshift(buffer,[-1 1]);
	   buffer(1,1) = output;   
	end
	
	res = buffer;
end
\end{lstlisting}
First the decoder will call analyze the input vector to return the syndrome value for input vector.
Then it will store the input vector in a buffer and calculate the length of the vector.

After initializing an empty error vector (same size as the input vector) it will for every symbol in the input vector do the following:
\begin{itemize}

\item Calculate the new syndrome and whether there's an error. Notice the first input is zero (explained in \fxnote{Insert reference}

\item Calculate the bit value modulo 2 of the buffer and the calculated error.

\item Shift all bits in the buffer and overwrite the 
\end{itemize}





\end{document}