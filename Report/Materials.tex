\documentclass[Main]{subfiles}

\begin{document}
\section{Materials and Methods}

(Teori om encoding og decoding)
\subsection{Encoding}
%(Matematisk eksempel)

\noindent Encoding for the code vector in systematic form:
\begin{equation}
X^{n-k}m(X) = q(X)g(X)+p(X)
\end{equation}

\noindent By adding the redundancy polynomial to the shifted message polynomial, the encoded vector in systematic form is obtained.
\begin{equation}
c(X) = X^{n-k}m(X)+p(X)
\end{equation}

\noindent A concrete example with the generator polynomial $g(X)=1+X^4+X^6+X^7+X^8$ is used to encode the following message polynomial:
$m(X)=1+X^2+X^4+X^6$. The code is a cyclic code, $C_{cyc}(15,7)$.

\noindent First the message is right shifted $n-k$ times.
\begin{equation}
X^{15-7}m(X) = X^8+X^{10}+X^{12}+X^{14}
\end{equation}

\noindent Find $p(X)$ by taking the remainder from $X^{n-k}m(x)$ divided by $g(X)$.

\begin{equation}
X^8
\end{equation}


\subsection{Decoding}

\end{document}
