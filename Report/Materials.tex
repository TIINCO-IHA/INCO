\documentclass[Main]{subfiles}

\begin{document}
\section*{Materials and Methods}

(Teori om encoding og decoding)
\subsection*{Encoding}
%(Matematisk eksempel)

\noindent Encoding for the code vector in systematic form:
\begin{equation}
X^{n-k}m(X) = q(X)g(X)+p(X)
\end{equation}

\noindent By adding the redundancy polynomial to the shifted message polynomial, the encoded vector in systematic form is obtained.
\begin{equation}
c(X) = X^{n-k}m(X)+p(X)
\end{equation}

\noindent A concrete example with the generator polynomial $g(X)=1+X^4+X^6+X^7+X^8$ is used to encode the following message polynomial:
$m(X)=1+X^2+X^4+X^6$. The code is a cyclic code, $C_{cyc}(15,7)$.

\noindent First the message is right shifted $n-k$ times.
\begin{equation}
X^{15-7}m(X) = X^8+X^{10}+X^{12}+X^{14}
\end{equation}

\noindent Find $p(X)$ by taking the remainder from $X^{n-k}m(x)$ divided by $g(X)$.

\begin{equation}
X^8
\end{equation}


\subsection*{Decoding}
In general when decoding a received vector the syndrome vector (S) is needed to find out if any error have occurred.
To find this the generator polynomial is used to make the generator matrix (G) which further is used to find the parity check matrix (H).
The formula to find the syndrome vector for a received vector (r) is:
\begin{equation}
S = r \circ H^T = (s_{0},s_{1}, ..., s_{n-k-1})
\end{equation}
To do all this by hand will look like the example below.
A generator polynomial could be $1+X+X^3$. 
The generator polynomial for this will be $G_{k \times n}$ where k = 4 and n = 7:
\begin{align*}
G =
\begin{bmatrix}
1 & 1 & 0 & 1 & 0 & 0 & 0\\
0 & 1 & 1 & 0 & 1 & 0 & 0\\
0 & 0 & 1 & 1 & 0 & 1 & 0\\
0 & 0 & 0 & 1 & 1 & 0 & 1\\
\end{bmatrix} \\
\end{align*}
To find the parity check matrix, the generator matrix need to be in systematic form.
This is done by creating a identity matrix at the end with detentions $(k \times k)$.
In systematic form the generator matrix is:
\begin{align*}
G =
\begin{bmatrix}
1 & 1 & 0 & 1 & 0 & 0 & 0\\
0 & 1 & 1 & 0 & 1 & 0 & 0\\
1 & 1 & 1 & 0 & 0 & 1 & 0\\
1 & 0 & 1 & 0 & 0 & 0 & 1\\
\end{bmatrix} \\
\end{align*}
The first n - k columns is the parity matrix.
The parity check matrix is in the form:
\begin{equation}
H = [I_{(n-k) \times (n-k)} \quad P^T_{(n-k) \times k}]
\end{equation}
\begin{align*}
H =
\begin{bmatrix}
1 & 0 & 0 & 1 & 0 & 1 & 1\\
0 & 1 & 0 & 1 & 1 & 1 & 0\\
0 & 0 & 1 & 0 & 1 & 1 & 1\\
\end{bmatrix} \\
\end{align*}







\end{document}