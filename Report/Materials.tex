\documentclass[Main]{subfiles}

\begin{document}
\section*{Materials and Methods}

(Teori om encoding og decoding)
\subsection{Encoding}
%(Matematisk eksempel)

\noindent Encoding for the code vector in systematic form:
\begin{equation}
X^{n-k}m(X) = q(X)g(X)+p(X)
\end{equation}

\noindent By adding the redundancy polynomial to the shifted message polynomial, the encoded vector in systematic form is obtained.
\begin{equation}
c(X) = X^{n-k}m(X)+p(X)
\end{equation}

\noindent A concrete example with the generator polynomial $g(X)=1+X^4+X^6+X^7+X^8$ is used to encode the following message polynomial:
$m(X)=1+X^2+X^4+X^6$, corresponding to $m=[1010101]$. The code is a cyclic code, $C_{cyc}(15,7)$.

\noindent First the message is right shifted $n-k$ times.
\begin{equation}
X^{15-7}m(X) = X^8+X^{10}+X^{12}+X^{14}
\end{equation}

\noindent Find $p(X)$ by taking the remainder from $X^{n-k}m(x)$ divided by $g(X)$.


\[
\renewcommand\arraystretch{1.2}
\begin{array}{cccccccccccccccccc}
& & & & & X^6 & + X^5 &  +X^4 & +X^2 & +1 & & & & & & & \\
\cline{6-10}
 X^8 & + X^7 & +X^6 & +X^4 & +1 \qquad \vline & X^{14} & +X^{12} & +X^{10} & + X^8 & & & & & & & & \\
& & & & & X^{14} & + X^{13}& + X^{12} & +X^{10} & +X^6 & & & & & & & \\
\cline{6-10}
& & & & & & X^{13} & & +X^8 & +X^6 & & & & & & &\\
&  &&    &&    &  X^{13} & +X^{12} & +X^{11} & +X^9 & +X^5 & \\
\cline{7-11}
& & &&  &&    &  X^{12} & +X^{11} & +X^9 & +X^8 & +X^6 & +X^5  \\
& & &&  &&    &  X^{12} & +X^{11} & +X^{10} & +X^8 & & +X^4  \\
\cline{8-13}
&  &&    &    &   & &  & X^{10} & +X^9 & +X^6 & +X^5 & +X^4 \\
&  &&    &    &   & &  & X^{10} & +X^9 & +X^8 & +X^6 & +X^2 \\
\cline{9-13}
&  &&    &    &   & & & & X^{8} & +X^5 & +X^4 & +X^2 \\
&  &&    &    &   & & & & X^{8} & +X^7 & +X^6 & +X^4 & +1 \\
\cline{10-14}
&  &&  & & & & & p(X) = & & X^{7} & +X^6 & +X^5 & +X^2 & +1 \\
\end{array}
\]

\noindent Now $c(X)$ can be calculated:

\begin{align}
c(X) = X^8m(X)+p(X) = 1 + X^2 + X^5 + X^6 + X^7 + X^8 + X^{10}+ X^{12} + X^{14} \\
c = [1 0 1 0 0 1 1 1 1 0 1 0 1 0 1]
\end{align}

\noindent It is easily seen that the encoded vector is in systematic form, hence the message vector corresponds to the last 7 bits.

\subsection{Decoding}

\end{document}
