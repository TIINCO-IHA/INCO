\documentclass[Main]{subfiles}

\begin{document}

\section{Conclusion}
In this mini-project an encoder and a Meggitt decoder for a $C_{cyc}(15,7)$ with the general polynomial $g(X)=1+X^4+X^6+X^7+X^8$ have been developed.
The encoder should generate a codeword from $g(X)$ and add 2 or less errors to to it.
The Meggitt decoder should be able to correct 2 or less errors and able to detect 3 or more errors.
\\
\\
The project have successfully made an encoder which is able to create a codeword with 3 or less errors.
It was also successful in making a Meggitt decoder which can  correct 2 or less errors.
The Meggitt decoder is able to detect most of the errors in codewords with 3 errors, but some times it correct them as 2 errors codeword.
As explained in the result, this could be because the codeword is the same as an other codeword with only 2 errors.
Therefore the Meggitt decode, do not know which codeword is the original and will handle it as the codeword with less errors.  
\\
\\
Overall the project have been successful.
The project is able of doing what was required and the members have learned a lot about encoding and Meggitt decoding.
It has been hard to learn the principle in the Meggitt decoder, but when this was all clear, it was easy to make the code and test it as well.
The hard part was to understand is the feedback in the syndrome register in the Meggitt decoder.
Which block register should have feedback?
In the example the delivered material about decoding of cyclic codes was implemented and extended to a 15 length codeword.
But some of the things that happened when initializing the syndrome register did not make sense and it was hard to figure out why.
When it was clear that the feedback in the syndrome register was depending on the general polynomial, it make a lot of sense and it was possible to make the code which gave the expected result.

\end{document}
